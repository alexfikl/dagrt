\documentclass[10pt]{article}

% Packages
\usepackage{amsmath}
\usepackage{amssymb}
\usepackage{bm}
\usepackage{comment}
\usepackage{enumerate}
\usepackage{fancyhdr}
\usepackage{pgf}
\usepackage[bf,small]{titlesec}
\usepackage{xcolor}

\renewcommand{\Re}{\mathop{\textrm{Re}}}

\begin{document}

% first part of first exercise 1 obviously has eigval 0 also also, if

\section*{Fastest-first forward Euler}

Assume that $\mathbf{y} = \begin{pmatrix} \mathbf{s}
  \\ \mathbf{f} \end{pmatrix}$ is divided into slow and fast
components. The initial value problem is
\newcommand{\slowfast}[1][]{\begin{pmatrix} \mathbf{s}_{#1}
\\ \mathbf{f}_{#1} \end{pmatrix}}
%
\begin{align*}
 \slowfast' &= \begin{pmatrix} A & B \\ C & D \end{pmatrix} \slowfast
 \\ \slowfast&(0) = \slowfast[0].
\end{align*}
%
For simplicity, we wish to integrate the fast component at a pace $h$
which is twice as fast as that of the slow component $2h$.
%
Use Euler's method to integrate this system, with the fast component
integrated first, to get
%
\begin{align*}
  \mathbf{f}_{n+1} &=
      \mathbf{f}_n + h(C \mathbf{s}_n + D \mathbf{f}_n) \\
  \mathbf{f}_{n+2} &=
      \mathbf{f}_{n+1} + h(C \mathbf{\tilde{s}}_{n+1} + D \mathbf{f}_{n+1}) \\
  \mathbf{s}_{n+2} &= \mathbf{s}_n + 2h(A \mathbf{s}_n + B \mathbf{f}_n)
\end{align*}
%
where $\mathbf{\tilde{s}}_{n+1}$ is a consistent approximation to
$\mathbf{s}(t_{n+1})$. For instance, we can take
$\mathbf{\tilde{s}}_{n+1} = \mathbf{s}_n$ (in this instance, calculating
$\mathbf{s}_{n+1}$ directly would not result in any work savings by the
multirate method).

Then the above scheme can be expressed as a linear system
%
\[ \slowfast[n+2] = G(h) \slowfast[n] \]
%
with
%
\[ G(h) = \begin{pmatrix} I + 2hA & 2hB \\
  2hC + h^2DC & I + 2hD + h^2D^2 \end{pmatrix}. \]
%
This formula for the step matrix (with a minor error) appears in Gear
and Wells.

\section*{Towards a stable timestep}

An indication that the timestep $h$ is \emph{stable} is when
%
\[ \rho(G(h)) \leq 1. \]
%
The goal of this section is to provide an analytical estimate for when
the above condition holds.

The approach taken by us is to write
%
\[ G(h) = I + hS(h) \]
%
where
%
\[ S(h) = \begin{pmatrix} 2A & 2B \\ 2C & 2D \end{pmatrix} +
          h \begin{pmatrix} 0 & 0 \\ DC & D^2 \end{pmatrix}. \]
%
Note that $I + hS(0)$ is equivalent to the step matrix for full forward
Euler run at a timestep of $2h$. It is well known that a stable timestep
$h$ then satisfies
%
\[ h \leq - \frac{2 \Re \lambda}{|\lambda|^2} \]
%
for all eigenvalues $\lambda$ of $S(0)$. In fact, this continues to hold
for $G(h) = I +hS(h)$.  Let $\lambda_h^{(i)}$ be the $i$th eigenvalue of
$S(h)$. Then a sufficient condition for $\rho(G(h)) \leq 1$ is
%
\begin{align*}
h \leq -\frac{2 \Re \lambda_h^{(i)}}{|\lambda_h^{(i)}|^2},
& \textrm{\quad for all } i.
\end{align*}
%
We can view $\lambda_h^{(i)}$ as a perturbation of the $i$th eigenvalue
of $S(0)$, i.e.
%
\[ \lambda_h^{(i)} = \lambda^{(i)} + \delta_h^{(i)}. \]
%
In fact it is possible to estimate $\delta_h^{(i)}$, and this can
motivate an estimate for a stable $h$.

\subsection*{Attempt 1: Bauer-Fike estimates}

By the Bauer-Fike theorem,
%
\[ | \delta_h^{(i)} | \leq h \, \kappa(X) \, \|E\|_2 \]
%
where $X$ diagonalizes $S(0)$, and $E$ is the perturbation
$\begin{pmatrix} 0 & 0 \\ DC & D^2 \end{pmatrix}$. Note that this
estimate is independent of the particular eigenvalue.

Let $\Delta_h$ be the estimate provided by the Bauer-Fike theorem. Given
that
%
\begin{align*}
|\lambda^{(i)} + \delta_h^{(i)}| &\leq |\lambda^{(i)}| + \Delta_h \\
\Re \lambda_i - \Delta_h &\leq \Re(\lambda^{(i)} + \delta_h^{(i)})
\end{align*}
%
it follows that if $h$ satisfies
%
\begin{align*}
h \leq -\frac{2(\Re \lambda^{(i)} - \Delta_h)}{(|\lambda^{(i)}| + \Delta_h)^2}
& \textrm{\quad for all } i
\end{align*}
%
then $h$ is stable. If we rewrite the above condition as a cubic
inequality
%
\[ \phi_i(h) = d^2 h^3 + 2l_idh^2 + (l_i^2 - 2d)h + 2r_i \leq 0 \]
%
\marginpar{\footnotesize What is the exact formula for the positive root
  of $\phi_i$?}
%
where
%
\begin{align*}
d &= \kappa(X) \|E\|_2 \\
l_i &= |\lambda^{(i)}| \\
r_i &= \Re \lambda^{(i)}
\end{align*}
%
we get the following theorem.
%
\newtheorem{thm}{Theorem}
\begin{thm}
Suppose that $\Re \lambda^{(i)} < 0$ for all $1 \leq i \leq n$. Then
$\phi_i$ has a single positive root $h_i^*$ and if
%
\[ 0 < h \leq \min_{1 \leq i \leq n} h_i^* \]
%
then $\rho(G(h)) \leq 1$. In particular, a stable $h > 0$ always exists.
\end{thm}

The uniqueness of the positive root is from Descarte's rule of
signs. The proof follows from $\phi_i(0) < 0$ using continuity.

Theorem 1 provides us with a means of computing a stable $h$. How close
is the estimate to maximal?

\subsection*{Accuracy of Bauer-Fike estimates}

In this section, we assume all eigenvalues are real. Let us work with
the system:
%
\begin{align*}
 \slowfast' &= \begin{pmatrix} A & B \\ 0 & D \end{pmatrix} \slowfast.
\end{align*}
%
This represents a limiting case where the fast component influences the
slow component but the fast component is permitted to evolve
independently. Further assume that $D$ has much larger eigenvalues than
$A$. This is a sensible assumption, since the fast component is meant to
evolve more quickly.

Now consider the multirate method applied to the above system
%
\[ \slowfast[n+2] = (I + hS(h)) \slowfast[n]. \]
%
In the analysis that led to the Bauer-Fike estimates, we started with
the point of view that our step matrix was a perturbation of $I +
hS(0)$. For purposes of this section, we will again wish to view the
multirate method as a ``perturbation'' of Euler's method. Note first if
we replace $S(h)$ with $S(0)$, we get Euler's method. The matrix $S(0)$
has the form
%
\[ S(0) = \begin{pmatrix} 2A & 2B \\ 0 & 2D \end{pmatrix}. \]
%

We first analyze the system
%
\[ \slowfast[n+2] = (I + hS(0)) \slowfast[n]. \]
%
The stability of the above single rate Euler method is relatively easy
to analyze.  The eigenvalues of a block triangular system are equal the
eigenvalues of the block diagonals. This allows us to derive a maximum
stable timestep
%
\[ h^* = -\frac{2 \Re 2\lambda^{(f)}}{|2 \lambda^{(f)}|^2} 
 = \frac{1}{| \lambda^{(f)} |}, \]
%
where is $\lambda^{(f)}$ the largest eigenvalue of $D$.  That is, we are
allowed to step the system synchronized at a rate up to that of $h^*$,
and this rate is controlled by the largest eigenvalue $\lambda^{(f)}$.

Note that this gives an insight to the multirate method's stability.
Intuitively, assuming the eigenvalues of $A$ are small enough, if we
decide to step $\mathbf{s}$ at a slower rate, \emph{this should not
  affect the stability of the overall system}. But stepping $\mathbf{s}$
at a slower rate, while maintaining the fast rate for $\mathbf{f}$, is
exactly the multirate method. So in the multirate case the stability is
therefore still controlled by the magnitude of $\lambda^{(f)}$, and we
have an estimate for the maximum stable (fast) timestep of the multirate
method, which is $h^*$.

Let us see how this compares to the stability estimate derived from the
Bauer-Fike theorem above. By above, this latter estimate $h$ satisfies
%
\begin{align*}
h = -\frac{2 (\Re 2\lambda^{(f)} - \kappa(X) \|D^2\|_2 h)}
{(|2\lambda^{(f)}| + \kappa(X) \|D^2\|_2 h)^2}
\end{align*}
%
where, as above, the factor of $2$ in front of $\lambda^{(f)}$ comes
from the fact that the eigenvalues of $S(0)$ are twice those of the
original system. By Gelfand's formula
%
\[ \lim_{n \to \infty} \|D^n\|^{1/n} = \rho(D) = |\lambda^{(f)}| \]
%
so it makes sense to write
%
\[
\|D^2\|_2 = M |\lambda^{(f)}|^2
\]
%
for some constant $M$ that is expected to be relatively small for
sufficiently regular $D$ (for instance, if $D$ is normal).
%
\marginpar{\footnotesize Can we give an estimate for $M$?}

Furthermore, noting that
%
\[ \Re 2 \lambda^{(f)} = - | 2 \lambda^{(f)} | \]
%
and rewriting and simplifying the above expression, we get

%
\[ h = \frac{1}{1 + \frac{1}{2} \kappa(X) M \, |\lambda^{(f)}| \, h} h^*. \]
%
That is, the Bauer-Fike timestep underestimates the stable fast timestep
by a factor of $1 + \frac{1}{2} \kappa(X) \, M \, |\lambda^{(f)}| \, h$.
\marginpar{\footnotesize Is this an upper or lower bound for the ratio
  of $h$ to $h^*$ in the general case, or neither?} In general, this can
be large. However, suppose $M, \lambda^{(f)}$ are small and $X$ is
well-conditioned. Then if the returned value of $h$ is itself small, the
estimate $h$ can be said to be within a small constant of the true
timestep $h^*$ for the above problem.

\subsection*{Stability study}

We experimentally studied the Bauer-Fike estimates on systems of the
form
%
\[ \slowfast' = X \begin{pmatrix} \lambda & 0 \\
  0 & n \lambda \end{pmatrix} X^{-1} \]
%
where
%
\[ X = \begin{pmatrix} \cos(\theta) & \cos(\theta + \beta) \\
       \sin(\theta) & \sin(\theta + \beta) \end{pmatrix} \]
%
and $\theta \in (0, \pi), \beta \in (0, \pi)$. We used $\lambda = -1$.

Table~\ref{table:h-to-computed} gives the result of computing the ratio
of the Bauer-Fike estimate versus the (computationally determined)
maximum stable timestep. The value given is the minimum ratio
encountered over all $\theta$ for the given combination of $n$ and
$\beta$.

\subsection*{Multirate efficiency}

The entries of table~\ref{table:h-to-unperturbed} compare the Bauer-Fike
estimate $h$ to the maximum stable timestep of the \emph{unperturbed
  (single-rate) method} $h^*$. In other words, $h^*$ is timestep that is
stable for Euler's method if the entire system is first multiplied by
the substep ratio. The data in table~\ref{table:h-to-unperturbed} gives
an idea of the \emph{efficiency} of the estimate: when switching from
single-rate to multi-rate, more timesteps need to be taken, but the work
at each timestep also decreases. At a certain point, running the
multirate method at a step $h$ becomes less efficient than running the
unperturbed method at a larger step $h^*$.

Ideally, we would like to have $h > wh^*$, where $w$ is the work-savings
of the multirate method compared to the single-rate method run at the
same timestep, and $h^*$ is the stable timestep of the unperturbed
single-rate method. If the work savings is estimated as the ratio of
right hand side evaluations, for forward Euler we have $w =
0.75$. Notice that the timestep given by the Bauer-Fike estimates may
not in general be efficient in this sense.

\begin{table}
\centering
\begin{tabular}{lllll}
$n$ & $\beta = \frac{\pi}{8}$ & $\beta = \frac{\pi}{4}$ &
  $\beta=\frac{3\pi}{8}$ & $\beta=\frac{\pi}{2}$ \\
\hline
\input{h-to-computed.tex}
\end{tabular}
\caption{Smallest ratio of estimate $h$ compared to maximum stable
  timestep}
\label{table:h-to-computed}}
\end{table}

\begin{table}
\centering
\begin{tabular}{lllll}
$n$ & $\beta = \frac{\pi}{8}$ & $\beta = \frac{\pi}{4}$ &
  $\beta=\frac{3\pi}{8}$ & $\beta=\frac{\pi}{2}$ \\
\hline
\input{h-to-unperturbed-ratios.tex}
\end{tabular}
\caption{Smallest ratio of estimate $h$ compared to unperturbed timestep}
\label{table:h-to-unperturbed}}
\end{table}

\end{document}
